%% Generated by Sphinx.
\def\sphinxdocclass{report}
\documentclass[letterpaper,10pt,english]{sphinxmanual}
\ifdefined\pdfpxdimen
   \let\sphinxpxdimen\pdfpxdimen\else\newdimen\sphinxpxdimen
\fi \sphinxpxdimen=.75bp\relax
\ifdefined\pdfimageresolution
    \pdfimageresolution= \numexpr \dimexpr1in\relax/\sphinxpxdimen\relax
\fi
%% let collapsible pdf bookmarks panel have high depth per default
\PassOptionsToPackage{bookmarksdepth=5}{hyperref}

\PassOptionsToPackage{warn}{textcomp}
\usepackage[utf8]{inputenc}
\ifdefined\DeclareUnicodeCharacter
% support both utf8 and utf8x syntaxes
  \ifdefined\DeclareUnicodeCharacterAsOptional
    \def\sphinxDUC#1{\DeclareUnicodeCharacter{"#1}}
  \else
    \let\sphinxDUC\DeclareUnicodeCharacter
  \fi
  \sphinxDUC{00A0}{\nobreakspace}
  \sphinxDUC{2500}{\sphinxunichar{2500}}
  \sphinxDUC{2502}{\sphinxunichar{2502}}
  \sphinxDUC{2514}{\sphinxunichar{2514}}
  \sphinxDUC{251C}{\sphinxunichar{251C}}
  \sphinxDUC{2572}{\textbackslash}
\fi
\usepackage{cmap}
\usepackage[T1]{fontenc}
\usepackage{amsmath,amssymb,amstext}
\usepackage{babel}



\usepackage{tgtermes}
\usepackage{tgheros}
\renewcommand{\ttdefault}{txtt}



\usepackage[Bjarne]{fncychap}
\usepackage[,maxlistdepth=10]{sphinx}

\fvset{fontsize=auto}
\usepackage{geometry}


% Include hyperref last.
\usepackage{hyperref}
% Fix anchor placement for figures with captions.
\usepackage{hypcap}% it must be loaded after hyperref.
% Set up styles of URL: it should be placed after hyperref.
\urlstyle{same}

\addto\captionsenglish{\renewcommand{\contentsname}{User Guide}}

\usepackage{sphinxmessages}
\setcounter{tocdepth}{2}



\title{New Horizons Mission Namespace}
\date{Jun 15, 2024}
\release{}
\author{NASA Planetary Data System}
\newcommand{\sphinxlogo}{\vbox{}}
\renewcommand{\releasename}{}
\makeindex
\begin{document}

\pagestyle{empty}
\sphinxmaketitle
\pagestyle{plain}
\sphinxtableofcontents
\pagestyle{normal}
\phantomsection\label{\detokenize{index::doc}}


\sphinxAtStartPar
The New Horizons Mission, the first to travel to the Pluto system, was launched
in 2006 and flew past Pluto on 14 July 2015. After the successful completion of
its primary mission, an extended mission was granted to investigate the Kuiper
Belt Object (KOB) 2014 MU69 (Arrokoth). The New Horizons mission data were
originally developed and archived in PDS3 format. This dictionary was developed
as part of the PDS4 migration effort, and includes all phases of the primary and
extended mission.

\sphinxstepscope


\chapter{Introduction}
\label{\detokenize{user/user-guide:introduction}}\label{\detokenize{user/user-guide::doc}}
\sphinxAtStartPar
This \sphinxstyleemphasis{User’s Guide} provides a brief overview of the
New Horizons Mission (NH or “nh:”) namespace for those working with data from
New Horizons primary or extended missions. The primary New Horizons mission
was to the Pluto system. The extended missions to date have been called
the “Kuiper Belt Extended Missions 1 and 2”, or “KEM1” and “KEM2”, in the
mission documentation and metadata.


\bigskip\hrule\bigskip

\begin{quote}

\sphinxAtStartPar
\sphinxstylestrong{Note} \sphinxstyleemphasis{that the New Horizons legacy data migration is in its early stages, with}
\sphinxstyleemphasis{labels being designed for each instrument in turn. This namespace is in active}
\sphinxstyleemphasis{development and will continue to be so for the forseeable future.}
\end{quote}


\bigskip\hrule\bigskip


\sphinxAtStartPar
Data from the primary and first extended (“KEM1”)
missions were archived in PDS3 format and migration is underway to convert the
legacy data into PDS4. These migrated products will also serve as templates for the
second extended (“KEM2”) mission, which will be delivered in PDS4 format.

\sphinxAtStartPar
This guide presents the major features of the namespace.


\chapter{Overview of the New Horizons (NH) Mission Dictionary}
\label{\detokenize{user/user-guide:overview-of-the-new-horizons-nh-mission-dictionary}}
\sphinxAtStartPar
The New Horizons Mission, the first to travel to the Pluto system, was launched
in 2006 and flew past Pluto on 14 July 2015. After the successful completion of
its primary mission, an extended mission was granted to investigate the Kuiper
Belt Object (KOB) 2014 MU69 (Arrokoth). The New Horizons mission data were
originally developed and archived in PDS3 format. This dictionary was developed
as part of the PDS4 migration effort, and includes all phases of the primary and
extended mission.
\begin{itemize}
\item {} 
\sphinxAtStartPar
\sphinxstylestrong{Primary Steward:} Anne Raugh, Small Bodies Node, University of Maryland (@acraugh on Github)

\item {} 
\sphinxAtStartPar
\sphinxstylestrong{Dictionary Repo:} \sphinxurl{https://github.com/pds-data-dictionaries/ldd-nh}

\item {} 
\sphinxAtStartPar
\sphinxstylestrong{Namespace Prefix:} nh:

\end{itemize}

\sphinxAtStartPar
Corrections, changes, and additions should be submitted through
the \sphinxhref{https://github.com/pds-data-dictionaries/PDS4-LDD-Issue-Repo}{PDS LDD Issue Repo}.


\chapter{Organization of Classes and Attributes}
\label{\detokenize{user/user-guide:organization-of-classes-and-attributes}}
\sphinxAtStartPar
The New Horizons dictionary has a single top\sphinxhyphen{}level class that must be used to
access any of the NH metadata classes. Below that, there are major subclasses
for metadata that is common to all (or multiple instruments), as well as
classes specific to particular instruments. Processed and calibrated data will
generally have additional classes to provide instrument\sphinxhyphen{}specific processing
details.

\sphinxAtStartPar
The following sections describe the major divisions of the New Horizons Mission
namespace, in the order in which they occur in the schema (and thus, labels).


\section{Top\sphinxhyphen{}Level Class: \textless{}nh:Mission\_Parameters\textgreater{}}
\label{\detokenize{user/user-guide:top-level-class-nh-mission-parameters}}
\sphinxAtStartPar
The \sphinxstyleemphasis{\textless{}nh:Mission\_Parameters\textgreater{}} class acts as a wrapper for all other NH classes.
It contains one required attribute and (as of this writing) two optional classes
for data specific to the Multispectral Visible Imaging Camera (MVIC) part of
in the Ralph instrument package.

\sphinxAtStartPar
The class contains a single required attribute, \sphinxstyleemphasis{\textless{}nh:mission\_phase\_name\textgreater{}}, with the
string identifying the mission phase. Mission phase names are unique to the
primary or extended mission in which they occur. Specifically, the phases in the
extended missions contain the extended mission acronym (“KEM1 Encounter”, for
example).

\sphinxAtStartPar
The major subclasses of the \sphinxstyleemphasis{\textless{}nh:Mission\_Parameters\textgreater{}} class are:
\begin{itemize}
\item {} 
\sphinxAtStartPar
{\hyperref[\detokenize{user/user-guide:observation-parameters}]{\sphinxcrossref{\DUrole{std,std-ref}{\textless{}nh:Observation\_Parameters\textgreater{}}}}}

\item {} 
\sphinxAtStartPar
{\hyperref[\detokenize{user/user-guide:mvic-calibration-information}]{\sphinxcrossref{\DUrole{std,std-ref}{\textless{}nh:MVIC\_Calibration\_Information\textgreater{}}}}}

\item {} 
\sphinxAtStartPar
{\hyperref[\detokenize{user/user-guide:radiometric-conversion-constants}]{\sphinxcrossref{\DUrole{std,std-ref}{\textless{}nh:Radiometric\_Converstion\_Constants\textgreater{}}}}}

\end{itemize}

\sphinxAtStartPar
You can see a complete outline of the namespace under the
{\hyperref[\detokenize{detailed/outline::doc}]{\sphinxcrossref{\DUrole{doc}{New Horizons Mission Namespace Outline}}}} topic.


\section{Subclass: \textless{}nh:Observation\_Parameters\textgreater{}}
\label{\detokenize{user/user-guide:subclass-nh-observation-parameters}}\label{\detokenize{user/user-guide:observation-parameters}}
\sphinxAtStartPar
The \sphinxstyleemphasis{\textless{}nh:Observation\_Parameters\textgreater{}} class provides details specific to the New
Horizons mission and the instrument used to make the observation comprising the
data product. It provides three attributes and two classes. As in the PDS
common namespace, in the NH dictionary attributes names are all lowercase;
class names are in title case.

\sphinxAtStartPar
This class contains:
\begin{itemize}
\item {} 
\sphinxAtStartPar
\textless{}nh:telemetry\_appid\textgreater{}

\item {} 
\sphinxAtStartPar
\textless{}nh:sequence\_id\textgreater{}

\item {} 
\sphinxAtStartPar
\textless{}nh:observation\_description\textgreater{}

\item {} 
\sphinxAtStartPar
\textless{}nh:Mission\_Elapsed\_Time\textgreater{}

\item {} 
\sphinxAtStartPar
\textless{}nh:Detector\textgreater{}

\item {} 
\sphinxAtStartPar
\textless{}nh:LORRI\_Target\_Information\textgreater{}

\item {} 
\sphinxAtStartPar
\textless{}nh:Spacecraft\_State\textgreater{}

\end{itemize}

\sphinxAtStartPar
None of these components is repeatable; all are expected to be present in all raw
and processed/calibrated data labels.
\begin{description}
\item[{\textless{}nh:telemetry\_appid\textgreater{}, \textless{}nh:sequence\_id\textgreater{}, and \textless{}nh:observation\_description\textgreater{}}] \leavevmode
\sphinxAtStartPar
These attributes are provided primarily for provenance and to provide some minimal
description of planned activities for the end user. The \sphinxstyleemphasis{nh:telementry\_appid} is
tied to instrument operating mode and to onboard processing like data compression.
The mission documentation for each instrument will provide further detail
if desired. The \sphinxstyleemphasis{\textless{}nh:sequence\_id\textgreater{}} ties into the instrument observing plan, and
the codes comprising that ID are roughly translated into something approaching
English in the \sphinxstyleemphasis{\textless{}nh:observation\_description\textgreater{}} string.

\item[{\textless{}nh:Mission\_Elapsed\_Time\textgreater{}}] \leavevmode
\sphinxAtStartPar
The \sphinxstyleemphasis{\textless{}nh:Mission\_Elapsed\_Time\textgreater{}} class provides the spacecraft clock partition and
count at the start and end of the observation comprising the data product.
The translation from spacecraft clock to UTC is dependent on the hardware and is
usually described in the mission documentation. Many missions and end\sphinxhyphen{}users use
the publicly available Navigation and Ancillary Information (NAIF) Toolkit to
perform this conversion.

\item[{\textless{}nh:Detector\textgreater{}}] \leavevmode
\sphinxAtStartPar
The \sphinxstyleemphasis{\textless{}nh:Detector\textgreater{}} class identifies the detector used to make the observation,
and includes classes to provide detector\sphinxhyphen{}specific parameters. “Detector” may
mean an instrument, or it may mean literally one of several detectors available
within an instrument (as is the case of the MVIC instrument, for example). This
class will contain detector\sphinxhyphen{}specific subclasses where needed to provide specific
observational settings for the detector.

\item[{\textless{}nh:LORRI\_Target\_Information\textgreater{}}] \leavevmode
\sphinxAtStartPar
The \sphinxstyleemphasis{\textless{}nh:LORRI\_Target\_Information\textgreater{}} class provides attributes specific to the
targeting of a LORRI imager observation.

\item[{\textless{}nh:Spacecraft\_State\textgreater{}}] \leavevmode
\sphinxAtStartPar
The \sphinxstyleemphasis{\textless{}nh:Spacecraft\_State\textgreater{}} class provides information about thruster firings,
spin state, scan rate, and spacecraft motion at the time of the observation.

\end{description}


\section{Subclass: \textless{}nh:MVIC\_Calibration\_Information\textgreater{}}
\label{\detokenize{user/user-guide:subclass-nh-mvic-calibration-information}}\label{\detokenize{user/user-guide:mvic-calibration-information}}
\sphinxAtStartPar
The \sphinxstyleemphasis{\textless{}nh:MVIC\_Calibration\_Information\textgreater{}} class is used in labels for processed
data from all seven MVIC detectors. It provides detector\sphinxhyphen{}specific quantities
used in processing the data, and in the case of the MVIC framing camers, it
provides the specific left\sphinxhyphen{} and right\sphinxhyphen{}side biases used to process each frame.

\sphinxAtStartPar
This class contains:
\begin{itemize}
\item {} 
\sphinxAtStartPar
\textless{}nh:physical\_pixel\_size\textgreater{}

\item {} 
\sphinxAtStartPar
\textless{}nh:read\_noise\textgreater{}

\item {} 
\sphinxAtStartPar
\textless{}nh:gain\textgreater{}

\item {} 
\sphinxAtStartPar
\textless{}nh:tdi\_median\_bias\_level\textgreater{}

\item {} 
\sphinxAtStartPar
\textless{}nh:Framing\_Biases\textgreater{}

\end{itemize}
\begin{description}
\item[{\textless{}nh:physical\_pixel\_size\textgreater{}, \textless{}nh:read\_noise\textgreater{} and \textless{}nh:gain\textgreater{}}] \leavevmode
\sphinxAtStartPar
The \sphinxstyleemphasis{\textless{}nh:physical\_pixel\_size\textgreater{}} value is constant for all pixels on all MVIC
detectors. It is provided explicitly for the convenience of users
further analyzing to the data. The \sphinxstyleemphasis{\textless{}nh:read\_noise\textgreater{}} and \sphinxstyleemphasis{\textless{}nh:gain\textgreater{}} are
also provided for all MVIC observations.

\item[{\textless{}nh:tdi\_median\_bias\_level\textgreater{}}] \leavevmode
\sphinxAtStartPar
The \sphinxstyleemphasis{\textless{}nh:tdi\_median\_bias\_level\textgreater{}} appears only in processed time delay integration
(TDI) observations, from the color channels and the two panchromatic TDI channels.
Bias levels for the TDI channels are determined during cruise operations and
may be updated through the course of the mission.

\item[{\textless{}nh:Framing\_Biases\textgreater{}}] \leavevmode
\sphinxAtStartPar
The \sphinxstyleemphasis{\textless{}nh:Framing\_Biases\textgreater{}} class only appears in processing sequences from the MVIC
framing array. It contains one \sphinxstyleemphasis{\textless{}nh:Frame\_Bias\_Levels\textgreater{}} class for each frame
comprising the observation that identifies the frame by number and lists the
left\sphinxhyphen{} and right\sphinxhyphen{}side bias levels applied
in processing that particular frame. For framing observations, bias is measured
during each observations using shielded pixels on either edge of the array.

\end{description}


\section{Subclass: \textless{}nh:Radiometric\_Conversion\_Constants\textgreater{}}
\label{\detokenize{user/user-guide:subclass-nh-radiometric-conversion-constants}}\label{\detokenize{user/user-guide:radiometric-conversion-constants}}
\sphinxAtStartPar
\sphinxstylestrong{NOTE:} \sphinxstyleemphasis{As of version 1.1.0, this class replaces the deprecated
\textless{}nh:MVIC\_Conversion\_Constants\textgreater{} class. The content of that class is included in this
one, with additional constants added as needed. This class is used by multiple instruments.}

\sphinxAtStartPar
The \sphinxstyleemphasis{\textless{}nh:Radiometric\_Conversion\_Constants\textgreater{}} class is used in labels for processed
data from all seven MVIC detectors. The MVIC pipeline does not produce “calibrated”
data in the sense that PDS defines “calibrated” \sphinxhyphen{} specifically, “Data reduced to
physical units”. The final reduction step depends on both the spectal characteristics
of the target and whether that target is resolved. Instead, the calibration
documentation provided with the archive includes formulae for applying the absolute
calibration for specific targets, and the constants needed to plug into the
formulae are provided in this class.

\sphinxAtStartPar
This class contains:
\begin{itemize}
\item {} 
\sphinxAtStartPar
\textless{}nh:pivot\_wavelength\textgreater{}

\item {} 
\sphinxAtStartPar
\textless{}nh:Resolved\_Source\textgreater{}

\item {} 
\sphinxAtStartPar
\textless{}nh:Unresolved\_Source\textgreater{}

\end{itemize}
\begin{description}
\item[{\textless{}nh:pivot\_wavelength\textgreater{}}] \leavevmode
\sphinxAtStartPar
The \sphinxstyleemphasis{\textless{}nh:pivot\_wavelength\textgreater{}} attribute contains the pivot wavelength of the
filter/dectector combination.

\item[{\textless{}nh:Resolved\_Source\textgreater{}}] \leavevmode
\sphinxAtStartPar
The \sphinxstyleemphasis{\textless{}nh:Resolved\_Source\textgreater{}} class provides the units of measure (units of radiance,
in the case of resolved targets) applicable to the resulting pixel values.
Other attributes contain the conversion constants for five targets:
\begin{itemize}
\item {} 
\sphinxAtStartPar
The Sun

\item {} 
\sphinxAtStartPar
Jupiter

\item {} 
\sphinxAtStartPar
(5145) Pholus, a centaur

\item {} 
\sphinxAtStartPar
Pluto

\item {} 
\sphinxAtStartPar
Charon

\item {} 
\sphinxAtStartPar
Arrokoth

\end{itemize}

\item[{\textless{}nh:Unresolved\_Source\textgreater{}}] \leavevmode
\sphinxAtStartPar
The \sphinxstyleemphasis{\textless{}nh:Unresolved\_Source\textgreater{}} class provides the units of measure (units of
irradiance, in the case of unresolved targets) applicable to the resulting pixel
values. Other attributes contain the conversion constants for five targets:
\begin{itemize}
\item {} 
\sphinxAtStartPar
The Sun

\item {} 
\sphinxAtStartPar
Jupiter

\item {} 
\sphinxAtStartPar
(5145) Pholus, a centaur

\item {} 
\sphinxAtStartPar
Pluto

\item {} 
\sphinxAtStartPar
Charon

\item {} 
\sphinxAtStartPar
Arrokoth

\end{itemize}

\end{description}

\sphinxstepscope


\chapter{New Horizons Mission Namespace Outline}
\label{\detokenize{detailed/outline:new-horizons-mission-namespace-outline}}\label{\detokenize{detailed/outline::doc}}
\sphinxAtStartPar
\sphinxstyleemphasis{\textless{}nh:Mission\_Parameters\textgreater{}} is the public entry point to the New Horizons Mission
namespace. This class contains all other NH classes and must be included to gain
access to them. Below is a summary outline of all classes and attributes
currently available in the NH mission dictionary, in the order in which they
would appear in a label if every single one was used.

\sphinxAtStartPar
Note that there are no real cases in which every single mission class and
attribute would appear in a single label. The point of this outline is primarily
to catalog what is present and show the required ordering within classes when
they are included in a label.

\begin{sphinxVerbatim}[commandchars=\\\{\}]
\PYG{o}{\PYGZlt{}}\PYG{n}{nh}\PYG{p}{:}\PYG{n}{Mission\PYGZus{}Parameters}\PYG{o}{\PYGZgt{}}
    \PYG{o}{\PYGZlt{}}\PYG{n}{nh}\PYG{p}{:}\PYG{n}{mission\PYGZus{}phase\PYGZus{}name}\PYG{o}{\PYGZgt{}}

    \PYG{o}{\PYGZlt{}}\PYG{n}{nh}\PYG{p}{:}\PYG{n}{Observation\PYGZus{}Parameters}\PYG{o}{\PYGZgt{}}
        \PYG{o}{\PYGZlt{}}\PYG{n}{nh}\PYG{p}{:}\PYG{n}{telemetry\PYGZus{}apid}\PYG{o}{\PYGZgt{}}
        \PYG{o}{\PYGZlt{}}\PYG{n}{nh}\PYG{p}{:}\PYG{n}{sequence\PYGZus{}id}\PYG{o}{\PYGZgt{}}
        \PYG{o}{\PYGZlt{}}\PYG{n}{nh}\PYG{p}{:}\PYG{n}{observation\PYGZus{}description}\PYG{o}{\PYGZgt{}}

        \PYG{o}{\PYGZlt{}}\PYG{n}{nh}\PYG{p}{:}\PYG{n}{Mission\PYGZus{}Elapsed\PYGZus{}Time}\PYG{o}{\PYGZgt{}}
            \PYG{o}{\PYGZlt{}}\PYG{n}{nh}\PYG{p}{:}\PYG{n}{clock\PYGZus{}parition}\PYG{o}{\PYGZgt{}}
            \PYG{o}{\PYGZlt{}}\PYG{n}{nh}\PYG{p}{:}\PYG{n}{start\PYGZus{}clock\PYGZus{}count}\PYG{o}{\PYGZgt{}}
            \PYG{o}{\PYGZlt{}}\PYG{n}{nh}\PYG{p}{:}\PYG{n}{stop\PYGZus{}clock\PYGZus{}count}\PYG{o}{\PYGZgt{}}

        \PYG{o}{\PYGZlt{}}\PYG{n}{nh}\PYG{p}{:}\PYG{n}{Detector}\PYG{o}{\PYGZgt{}}
            \PYG{o}{\PYGZlt{}}\PYG{n}{nh}\PYG{p}{:}\PYG{n}{detector\PYGZus{}name}\PYG{o}{\PYGZgt{}}
            \PYG{o}{\PYGZlt{}}\PYG{n}{nh}\PYG{p}{:}\PYG{n}{detector\PYGZus{}type}\PYG{o}{\PYGZgt{}}

            \PYG{o}{\PYGZlt{}}\PYG{n}{nh}\PYG{p}{:}\PYG{n}{Alice\PYGZus{}Details}\PYG{o}{\PYGZgt{}}
                \PYG{o}{\PYGZlt{}}\PYG{n}{nh}\PYG{p}{:}\PYG{n}{aperture}\PYG{o}{\PYGZgt{}}

            \PYG{o}{\PYGZlt{}}\PYG{n}{nh}\PYG{p}{:}\PYG{n}{Ralph\PYGZus{}Details}\PYG{o}{\PYGZgt{}}
                \PYG{o}{\PYGZlt{}}\PYG{n}{nh}\PYG{p}{:}\PYG{n}{met510}\PYG{o}{\PYGZgt{}}
                \PYG{o}{\PYGZlt{}}\PYG{n}{nh}\PYG{p}{:}\PYG{n}{hk\PYGZus{}packet\PYGZus{}is\PYGZus{}real}\PYG{o}{\PYGZgt{}}

            \PYG{o}{\PYGZlt{}}\PYG{n}{nh}\PYG{p}{:}\PYG{n}{LEISA\PYGZus{}Details}\PYG{o}{\PYGZgt{}}
                \PYG{o}{\PYGZlt{}}\PYG{n}{nh}\PYG{p}{:}\PYG{n}{scan\PYGZus{}type}\PYG{o}{\PYGZgt{}}
                \PYG{o}{\PYGZlt{}}\PYG{n}{nh}\PYG{p}{:}\PYG{n}{leisa\PYGZus{}mode}\PYG{o}{\PYGZgt{}}
                \PYG{o}{\PYGZlt{}}\PYG{n}{nh}\PYG{p}{:}\PYG{n}{leisa\PYGZus{}offset\PYGZus{}1}\PYG{o}{\PYGZgt{}}
                \PYG{o}{\PYGZlt{}}\PYG{n}{nh}\PYG{p}{:}\PYG{n}{leisa\PYGZus{}offset\PYGZus{}2}\PYG{o}{\PYGZgt{}}
                \PYG{o}{\PYGZlt{}}\PYG{n}{nh}\PYG{p}{:}\PYG{n}{leisa\PYGZus{}offset\PYGZus{}3}\PYG{o}{\PYGZgt{}}
                \PYG{o}{\PYGZlt{}}\PYG{n}{nh}\PYG{p}{:}\PYG{n}{leisa\PYGZus{}offset\PYGZus{}4}\PYG{o}{\PYGZgt{}}
                \PYG{o}{\PYGZlt{}}\PYG{n}{nh}\PYG{p}{:}\PYG{n}{leisa\PYGZus{}rate}\PYG{o}{\PYGZgt{}}
                \PYG{o}{\PYGZlt{}}\PYG{n}{nh}\PYG{p}{:}\PYG{n}{leisa\PYGZus{}side}\PYG{o}{\PYGZgt{}}
                \PYG{o}{\PYGZlt{}}\PYG{n}{nh}\PYG{p}{:}\PYG{n}{leisa\PYGZus{}temperature}\PYG{o}{\PYGZgt{}}

            \PYG{o}{\PYGZlt{}}\PYG{n}{nh}\PYG{p}{:}\PYG{n}{LORRI\PYGZus{}Details}\PYG{o}{\PYGZgt{}}
                \PYG{o}{\PYGZlt{}}\PYG{n}{nh}\PYG{p}{:}\PYG{n}{binning\PYGZus{}mode}\PYG{o}{\PYGZgt{}}

            \PYG{o}{\PYGZlt{}}\PYG{n}{nh}\PYG{p}{:}\PYG{n}{MVIC\PYGZus{}Details}\PYG{o}{\PYGZgt{}}
                \PYG{o}{\PYGZlt{}}\PYG{n}{nh}\PYG{p}{:}\PYG{n}{scan\PYGZus{}type}\PYG{o}{\PYGZgt{}}
                \PYG{o}{\PYGZlt{}}\PYG{n}{nh}\PYG{p}{:}\PYG{n}{tdi\PYGZus{}rate}\PYG{o}{\PYGZgt{}}

            \PYG{o}{\PYGZlt{}}\PYG{n}{nh}\PYG{p}{:}\PYG{n}{SWAP\PYGZus{}Details}\PYG{o}{\PYGZgt{}}
                \PYG{o}{\PYGZlt{}}\PYG{n}{nh}\PYG{p}{:}\PYG{n}{sweep\PYGZus{}samples\PYGZus{}count}\PYG{o}{\PYGZgt{}}

        \PYG{o}{\PYGZlt{}}\PYG{n}{nh}\PYG{p}{:}\PYG{n}{LORRI\PYGZus{}Target\PYGZus{}Information}\PYG{o}{\PYGZgt{}}
            \PYG{o}{\PYGZlt{}}\PYG{n}{nh}\PYG{p}{:}\PYG{n}{approx\PYGZus{}target\PYGZus{}name}\PYG{o}{\PYGZgt{}}
            \PYG{o}{\PYGZlt{}}\PYG{n}{nh}\PYG{p}{:}\PYG{n}{approx\PYGZus{}target\PYGZus{}line}\PYG{o}{\PYGZgt{}}
            \PYG{o}{\PYGZlt{}}\PYG{n}{nh}\PYG{p}{:}\PYG{n}{approx\PYGZus{}target\PYGZus{}sample}\PYG{o}{\PYGZgt{}}

        \PYG{o}{\PYGZlt{}}\PYG{n}{nh}\PYG{p}{:}\PYG{n}{Spacecraft\PYGZus{}State}\PYG{o}{\PYGZgt{}}
            \PYG{o}{\PYGZlt{}}\PYG{n}{nh}\PYG{p}{:}\PYG{n}{thruster\PYGZus{}x\PYGZus{}enabled}\PYG{o}{\PYGZgt{}}
            \PYG{o}{\PYGZlt{}}\PYG{n}{nh}\PYG{p}{:}\PYG{n}{thruster\PYGZus{}y\PYGZus{}enabled}\PYG{o}{\PYGZgt{}}
            \PYG{o}{\PYGZlt{}}\PYG{n}{nh}\PYG{p}{:}\PYG{n}{thruster\PYGZus{}z\PYGZus{}enabled}\PYG{o}{\PYGZgt{}}
            \PYG{o}{\PYGZlt{}}\PYG{n}{nh}\PYG{p}{:}\PYG{n}{gc\PYGZus{}scan\PYGZus{}rate}\PYG{o}{\PYGZgt{}}
            \PYG{o}{\PYGZlt{}}\PYG{n}{nh}\PYG{p}{:}\PYG{n}{target\PYGZus{}motion\PYGZus{}rate}\PYG{o}{\PYGZgt{}}
            \PYG{o}{\PYGZlt{}}\PYG{n}{nh}\PYG{p}{:}\PYG{n}{relative\PYGZus{}control\PYGZus{}mode\PYGZus{}active}\PYG{o}{\PYGZgt{}}
            \PYG{o}{\PYGZlt{}}\PYG{n}{nh}\PYG{p}{:}\PYG{n}{pointing\PYGZus{}method}\PYG{o}{\PYGZgt{}}
            \PYG{o}{\PYGZlt{}}\PYG{n}{nh}\PYG{p}{:}\PYG{n}{spacecraft\PYGZus{}spin\PYGZus{}state}\PYG{o}{\PYGZgt{}}

    \PYG{o}{\PYGZlt{}}\PYG{n}{nh}\PYG{p}{:}\PYG{n}{MVIC\PYGZus{}Calibration\PYGZus{}Information}\PYG{o}{\PYGZgt{}}
        \PYG{o}{\PYGZlt{}}\PYG{n}{nh}\PYG{p}{:}\PYG{n}{physical\PYGZus{}pixel\PYGZus{}size}\PYG{o}{\PYGZgt{}}
        \PYG{o}{\PYGZlt{}}\PYG{n}{nh}\PYG{p}{:}\PYG{n}{read\PYGZus{}noise}\PYG{o}{\PYGZgt{}}
        \PYG{o}{\PYGZlt{}}\PYG{n}{nh}\PYG{p}{:}\PYG{n}{gain}\PYG{o}{\PYGZgt{}}
        \PYG{o}{\PYGZlt{}}\PYG{n}{nh}\PYG{p}{:}\PYG{n}{tdi\PYGZus{}median\PYGZus{}bias\PYGZus{}level}\PYG{o}{\PYGZgt{}}

        \PYG{o}{\PYGZlt{}}\PYG{n}{nh}\PYG{p}{:}\PYG{n}{Framing\PYGZus{}Biases}\PYG{o}{\PYGZgt{}}
            \PYG{o}{\PYGZlt{}}\PYG{n}{nh}\PYG{p}{:}\PYG{n}{Frame\PYGZus{}Bias\PYGZus{}Levels}\PYG{o}{\PYGZgt{}}
                \PYG{o}{\PYGZlt{}}\PYG{n}{nh}\PYG{p}{:}\PYG{n}{frame\PYGZus{}number}\PYG{o}{\PYGZgt{}}
                \PYG{o}{\PYGZlt{}}\PYG{n}{nh}\PYG{p}{:}\PYG{n}{left\PYGZus{}side\PYGZus{}median\PYGZus{}bias}\PYG{o}{\PYGZgt{}}
                \PYG{o}{\PYGZlt{}}\PYG{n}{nh}\PYG{p}{:}\PYG{n}{right\PYGZus{}side\PYGZus{}median\PYGZus{}bias}\PYG{o}{\PYGZgt{}}

    \PYG{o}{\PYGZlt{}}\PYG{n}{nh}\PYG{p}{:}\PYG{n}{Radiometric\PYGZus{}Conversion\PYGZus{}Constants}\PYG{o}{\PYGZgt{}}
        \PYG{o}{\PYGZlt{}}\PYG{n}{nh}\PYG{p}{:}\PYG{n}{pivot\PYGZus{}wavelength}\PYG{o}{\PYGZgt{}}

        \PYG{o}{\PYGZlt{}}\PYG{n}{nh}\PYG{p}{:}\PYG{n}{Resolved\PYGZus{}Source}\PYG{o}{\PYGZgt{}}
            \PYG{o}{\PYGZlt{}}\PYG{n}{nh}\PYG{p}{:}\PYG{n}{units\PYGZus{}of\PYGZus{}conversion\PYGZus{}constants}\PYG{o}{\PYGZgt{}}
            \PYG{o}{\PYGZlt{}}\PYG{n}{nh}\PYG{p}{:}\PYG{n}{solar\PYGZus{}constant}\PYG{o}{\PYGZgt{}}
            \PYG{o}{\PYGZlt{}}\PYG{n}{nh}\PYG{p}{:}\PYG{n}{jupiter\PYGZus{}constant}\PYG{o}{\PYGZgt{}}
            \PYG{o}{\PYGZlt{}}\PYG{n}{nh}\PYG{p}{:}\PYG{n}{pholus\PYGZus{}constant}\PYG{o}{\PYGZgt{}}
            \PYG{o}{\PYGZlt{}}\PYG{n}{nh}\PYG{p}{:}\PYG{n}{pluto\PYGZus{}constant}\PYG{o}{\PYGZgt{}}
            \PYG{o}{\PYGZlt{}}\PYG{n}{nh}\PYG{p}{:}\PYG{n}{charon\PYGZus{}constant}\PYG{o}{\PYGZgt{}}
            \PYG{o}{\PYGZlt{}}\PYG{n}{nh}\PYG{p}{:}\PYG{n}{arrokoth\PYGZus{}constant}\PYG{o}{\PYGZgt{}}

        \PYG{o}{\PYGZlt{}}\PYG{n}{nh}\PYG{p}{:}\PYG{n}{Unresolved\PYGZus{}Source}\PYG{o}{\PYGZgt{}}
            \PYG{o}{\PYGZlt{}}\PYG{n}{nh}\PYG{p}{:}\PYG{n}{units\PYGZus{}of\PYGZus{}conversion\PYGZus{}constants}\PYG{o}{\PYGZgt{}}
            \PYG{o}{\PYGZlt{}}\PYG{n}{nh}\PYG{p}{:}\PYG{n}{solar\PYGZus{}constant}\PYG{o}{\PYGZgt{}}
            \PYG{o}{\PYGZlt{}}\PYG{n}{nh}\PYG{p}{:}\PYG{n}{jupiter\PYGZus{}constant}\PYG{o}{\PYGZgt{}}
            \PYG{o}{\PYGZlt{}}\PYG{n}{nh}\PYG{p}{:}\PYG{n}{pholus\PYGZus{}constant}\PYG{o}{\PYGZgt{}}
            \PYG{o}{\PYGZlt{}}\PYG{n}{nh}\PYG{p}{:}\PYG{n}{pluto\PYGZus{}constant}\PYG{o}{\PYGZgt{}}
            \PYG{o}{\PYGZlt{}}\PYG{n}{nh}\PYG{p}{:}\PYG{n}{charon\PYGZus{}constant}\PYG{o}{\PYGZgt{}}
            \PYG{o}{\PYGZlt{}}\PYG{n}{nh}\PYG{p}{:}\PYG{n}{arrokoth\PYGZus{}constant}\PYG{o}{\PYGZgt{}}
\end{sphinxVerbatim}

\sphinxstepscope


\chapter{Alphabetical List of Classes}
\label{\detokenize{detailed/classList:alphabetical-list-of-classes}}\label{\detokenize{detailed/classList::doc}}
\sphinxAtStartPar
A complete list of all classes in the New Horizons Mission Dictionary, in alphabetical
order, is available through the \sphinxhref{https://pds.nasa.gov/datastandards/documents/dd/current/PDS4\_PDS\_DD\_1L00/webhelp/all/}{PDS4 Data Dictionary}
page, which is regenerated automatically with each release of the PDS4 Information Model.

\sphinxAtStartPar
To find the New Horizons Mission class list, look down the list of (alphabetically sorted)
dictionary prefixes in the left menu for “Classes in the nh namespace”.
Select that item and the list of classes will be presented on both the left and
the right as clickable links.

\sphinxAtStartPar
Clicking on the specific class name will produce a grid with the full, formal
definition of the class.

\sphinxAtStartPar
Clicking on the class name in the “Referenced from:” line at the bottom of the
grid will take you to the containing class, where you can see the cardinality
of the class (i.e., whether it is required, optional, or repeatable) in the
containing class.

\sphinxAtStartPar
You can also click on the attribute names listed to see details of the
attribute definitions.

\sphinxstepscope


\chapter{Alphabetical List of Attributes}
\label{\detokenize{detailed/attributeList:alphabetical-list-of-attributes}}\label{\detokenize{detailed/attributeList::doc}}
\sphinxAtStartPar
A complete list of all attributes in the New Horizons Mission Dictionary, in alphabetical
order, is available through the \sphinxhref{https://pds.nasa.gov/datastandards/documents/dd/current/PDS4\_PDS\_DD\_1L00/webhelp/all/}{PDS4 Data Dictionary}
page, which is regenerated automatically with each release of the PDS4 Information Model.

\sphinxAtStartPar
To find the New Horizons Mission attribute list, look down the list of (alphabetically sorted)
dictionary prefixes in the left menu for “Attributes in the nh namespace”.
Select that item and the list of attributes will be presented on both the left and the
right as clickable links.

\sphinxAtStartPar
Clicking on the specific attribute name will produce a grid with the full, formal
definition of the attribute, including data type, restrictions on values, and the
list of defined permissible values (if any) and their definitions.

\sphinxAtStartPar
\sphinxstylestrong{Note} that attributes might appear as members of different classes, and that their
definitions, or more likely their permissible values, might be context\sphinxhyphen{}dependent.

\sphinxAtStartPar
Clicking on the class name in the title bar of the attribute grid will take you to
the definition of the class containing that attribute.

\sphinxAtStartPar
If the attribute has an associated unit of measure type, that attribute \sphinxstyleemphasis{must} have
an XML attribute called “unit” in its tag when it is used. For example:

\begin{sphinxVerbatim}[commandchars=\\\{\}]
\PYG{o}{\PYGZlt{}}\PYG{n}{nh}\PYG{p}{:}\PYG{n}{tdi\PYGZus{}rate} \PYG{n}{unit}\PYG{o}{=}\PYG{l+s+s2}{\PYGZdq{}}\PYG{l+s+s2}{Hz}\PYG{l+s+s2}{\PYGZdq{}}\PYG{o}{\PYGZgt{}}\PYG{l+m+mf}{40.4694}\PYG{o}{\PYGZlt{}}\PYG{o}{/}\PYG{n}{nh}\PYG{p}{:}\PYG{n}{tdi\PYGZus{}rate}\PYG{o}{\PYGZgt{}}
\end{sphinxVerbatim}

\sphinxAtStartPar
You can see valid values to use for the “unit=” XML attribute by clicking on the
value of “Unit of Measure Type” in the grid.

\sphinxstepscope


\chapter{Mockup Labels}
\label{\detokenize{examples/examples:mockup-labels}}\label{\detokenize{examples/examples::doc}}
\sphinxAtStartPar
Mockup labels are created during the initial design and test phase for
each instrument. They are not particularly good templates for designing
a \sphinxstyleemphasis{real} label, in that they usually omit the half dosen or so discipline
dictionaries needed to provide complete metadata for an actual observational
product. But they do so plausible values for the specific classes illustrated,
as found in a sample product from the mission.

\sphinxAtStartPar
Mockup labels can be found in the New Horizions Mission Dictionary
GitHub repo: \sphinxurl{https://github.com/pds-data-dictionaries/ldd-nh}. They are in the
\sphinxstyleemphasis{test/examples/} directory. Available mockups are:
\begin{itemize}
\item {} 
\sphinxAtStartPar
MVIC Red Channel Processed Data (\sphinxhref{https://github.com/pds-data-dictionaries/ldd-nh/tree/main/test/examples/MVICmc0Proc\_VALID.xml}{MVICred})

\item {} 
\sphinxAtStartPar
MVIC Framing Array Processed Data (\sphinxhref{https://github.com/pds-data-dictionaries/ldd-nh/tree/main/test/examples/MVICmpfProc\_VALID.xml}{MVICframe})

\item {} 
\sphinxAtStartPar
LEISA Processed Data (\sphinxhref{https://github.com/pds-data-dictionaries/ldd-nh/tree/main/test/examples/LEISAProc\_VALID.xml}{LEISA})

\item {} 
\sphinxAtStartPar
SWAP Histogram Data (\sphinxhref{https://github.com/pds-data-dictionaries/ldd-nh/tree/main/test/examples/SWAPhistProc\_VALID.xml}{SWAPhist})

\end{itemize}


\chapter{Instrument Class Summaries}
\label{\detokenize{examples/examples:instrument-class-summaries}}
\sphinxAtStartPar
These abbreviated labels, also available in the \sphinxhref{https://github.com/pds-data-dictionaries/ldd-nh}{GitHub repo} \sphinxstyleemphasis{test/examples} directory, show only
the \textless{}nh:Mission\_Parameters\textgreater{} class with the class(es) specific to the instrument.
The values shown in the labels are nonsensical.
\begin{itemize}
\item {} 
\sphinxAtStartPar
\sphinxhref{https://github.com/pds-data-dictionaries/ldd-nh/tree/main/test/examples/AliceClasses\_VALID.xml}{Alice Classes}

\item {} 
\sphinxAtStartPar
\sphinxhref{https://github.com/pds-data-dictionaries/ldd-nh/tree/main/test/examples/LEISAClasses\_VALID.xml}{LEISA Classes}

\item {} 
\sphinxAtStartPar
\sphinxhref{https://github.com/pds-data-dictionaries/ldd-nh/tree/main/test/examples/LORRIClasses\_VALID.xml}{LORRI Classes}

\end{itemize}



\renewcommand{\indexname}{Index}
\printindex
\end{document}